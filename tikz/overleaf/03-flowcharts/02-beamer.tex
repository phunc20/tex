\documentclass{beamer}
\usecolortheme{owl}
\usepackage{tikz}
\usetikzlibrary{shapes.geometric, arrows}


\begin{document}

\tikzstyle{startend} = [rectangle, rounded corners, minimum width=3cm, minimum height=1cm,
                         text centered, draw=green, very thick, fill=red!30]
\tikzstyle{io} = [trapezium,
                  trapezium left angle=70,
                  trapezium right angle=110,
                  minimum width=3cm,
                  minimum height=1cm,
                  text centered,
                  very thick,
                  draw=white,
                  fill=blue!30
]
\tikzstyle{process} = [rectangle,
                       minimum width=3cm,
                       minimum height=1cm,
                       text centered,
                       text width=3cm,
                       very thick,
                       draw=white,
                       fill=orange!30
]
\tikzstyle{decision} = [diamond,
                        minimum width=3cm,
                        minimum height=1cm,
                        text centered,
                        very thick,
                        draw=white,
                        fill=green!30
]

\tikzstyle{arrow} = [thick,
                     ->,
                     color=white,
                     >=stealth,
]



\begin{frame}
  \begin{tikzpicture}[node distance=2cm]
    % label for future ref.
    % what   label     style                                      displayed_text
    \node    (start)   [startend]                                 {Anfang};
    \node    (in)      [io, below of=start]                       {Eingabe};
    \node    (proc1)   [process, below of=in]                     {Prozess 1};
    \node    (dec1)    [decision, right of=proc1, xshift=2cm]     {Entscheidung};
    \node    (proc2b)  [process, above of=dec1, yshift=1cm]       {Prozess 2b: wortreich, wortreich, wortreich};
    \node    (proc2a)  [process, right of=dec1, xshift=2cm]       {Prozess 2a};
    \node    (out)     [io, above of=proc2a]                      {Ausgabe};
    \node    (end)    [startend, above of=out]                    {Zweck};

    % what   style     start_pt   end_pt
    \draw    [arrow]   (start) -- (in);
    \draw    [arrow]   (in) -- (proc1);
    \draw    [arrow]   (proc1) -- (dec1);
    \draw    [arrow]   (dec1) -- node[anchor=south] {Ja} (proc2a);
    \draw    [arrow]   (dec1) -- node[anchor=west] {Nein} (proc2b);
    \draw    [arrow]   (proc2b) -- (proc1);
    \draw    [arrow]   (proc2a) -- (out);
    \draw    [arrow]   (out) -- (end);
  \end{tikzpicture}
\end{frame}

\end{document}
