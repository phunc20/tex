\documentclass{article}
\usepackage{multirow}  % in order to use \multirow; \mulitcolumn is available by default


\begin{document}
% 6.1 Laisser une case vide en haut et à gauche d'un tableau
\begin{tabular}{|l|c|c|}
  % You might wonder why we need to use \multicolumn
  % in the 1st row. Let's find that out.
  \cline{2-3} {} & Paris & Oslo \\
  \hline
  Rome & 1447 Km & 2565 km \\
  \hline
  Prague & 1061 Km & 1202 km \\
  \hline
\end{tabular}

\vspace{3cm}
\begin{tabular}{|l|c|c|}
  \cline{2-3} \multicolumn{1}{l|}{} & Paris & Oslo \\
  \hline
  Rome & 1447 Km & 2565 km \\
  \hline
  Prague & 1061 Km & 1202 km \\
  \hline
\end{tabular}


% 6.2 Changer l'alignement d'une seule cellule
\vspace{3cm}
\begin{tabular}{|l|r|}
  \hline
  %\multicolumn{2}{c|c}{Pays & Monnaie} \\  % This won't work.
  \multicolumn{1}{|c|}{Pays} & \multicolumn{1}{c|}{Monnaie} \\
  \hline
  Japon & Yen \\
  Russie & Rouble \\
  Zimbabwe & Dollar zimbabwéen \\
  \hline
\end{tabular}


% 6.3 Combiner lignes et colonnes multiples
\vspace{3cm}
\begin{tabular}{|c|c|c|c|}
  \hline
  1 & 2 & 3 & 4 \\
  \hline
  12 & \multicolumn{2}{c|}{\multirow{2}*{2007}} & 5 \\
  \cline{1-1} \cline{4-4}
  %11 & & & 6 \\  % This will be wrong.
  11 & \multicolumn{2}{c|}{} & 6 \\
  \hline
  10 & 9 & 8 & 7 \\
  \hline
\end{tabular}



\end{document}
