\documentclass[dvips]{beamer}
\usepackage[T1]{fontenc}
\usepackage[latin1]{inputenc}
\usepackage[francais]{babel}

% choix du theme
\usetheme{Warsaw}
\mode<presentation>
% supprime les symboles de navigation
\setbeamertemplate{navigation symbols}{}
% fait apparaitre en grise les animations
\setbeamercovered{transparent}

\title{Mon titre}
\author{L'auteur}
\begin{document}

\begin{frame}
  % Equivalent de \maketitle
  \titlepage
\end{frame}

\begin{frame}
  \frametitle{Table des matières}
  \tableofcontents
\end{frame}

\section{Première partie}
\begin{frame}
  \frametitle{Le titre du premier transparent}
  \framesubtitle{Eventuellement un sous-titre}

  Et du contenu, que l'on peut
  \alert{souligner}.
  \uncover<2-3>{Ceci n'est révélé que
  dans les deux derniers affichages.}

  \begin{itemize}
    \item<1-> J'apparais en premier;
    \item<2-> moi en deuxième;
    \item<3-> et moi en dernier;
  \end{itemize}

  \begin{block}{Attention}
    Du texte mis en évidence
  \end{block}
\end{frame}

\section{Seconde partie}
\begin{frame}
  \frametitle{Le nouveau titire}
  Et du contenu...
\end{frame}

\end{document}
